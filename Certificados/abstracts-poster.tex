\setcounter{figure}{0}
\begin{center}
%{\huge \textbf{A. García-Abuín}}
\section*{}
%\section*{P-1~~~~ A. García Abuín}
\addcontentsline{toc}{section}{P-1~~~~ A. García Abuín}
{\bf \Large
Surface behavior of N-methyl-2-pyrrolidone and
alkanolamines mixtures
}
\\
\vspace{0.5cm}
\underline{A. García-Abuín}$^{1}$, D. Gómez-Díaz$^{1}$, A. B. López$^{2}$, and J. M. Navaza$^{1}$
\\
\vspace{0.5cm}
{\it 
$^{1}$ Dept. of Chemical Engineering. ETSE. University of Santiago de Compostela. Rúa Lope Gómez de
Marzoa s/n. E-15706. Santiago de Compostela. Spain

$^{2}$ Dept. of Chemical, Environmental and Materials Engineering. EPS. University of Jaén. Paraje Las
Lagunillas s/n. E-23071. Jaén. Spain
}
\\
\vspace{0.5cm}
{\it E-mail: alicia.garcia@rai.usc.es}
\\
\vspace{0.5cm}
\end{center}
Cyclic amides have shown important characteristics such as high density,
high boiling point, and high polarity solvents, which allow the usability at industrial
level. Also the high solubility in water allows the use of this kind of substance in a
wide range of industrial and laboratory operations. More specifically N-methyl-2pyrrolidone
(NMP) has shown selectivity regards unsaturated and aromatic
hydrocarbons and sulphur gases. The low reactivity and high solubility are important
characteristics that allow use the NMP for extraction agent in the lubricant oil
processing and the scrubbing treatment of natural gas [1]. On the other hand the
excellent thermal and chemical stability are another interesting characteristic for the
use of NMP as solvent in different reaction systems. Also the NMP could be used as
co-solvent with water, hydrocarbons, alcohols, glycol ether and ketones [2]. NMP in
aqueous solution is used to carbon dioxide capture process by means of physical
absorption [3]. The knowledge of different physical properties of systems that
involve NMP is an important starting point to optimize different mass transfer
operations.

Present work analyses binary mixtures formed by N-methyl-2-pyrrolidone
and different amines such as: diethanolamine (DEA), triethanolamine (TEA),
1-amino-2-propanol (MIPA) and bis(2-hydroxypropyl)amine (DIPA). The surface
tension value has been obtained over the entire composition range and for
temperatures from 20 to 50ºC. Systems that use DIPA the composition range has
been reduced taken into account the melt temperature of this compound.

The surface tension was determined by employing a Krüss K-11 tensiometer
using the Wilhelmy plate method. The plate employed was a commercial platinum
plate supplied by Krüss. The platinum plate was cleaned and flame dried before each
measurement. Each surface tension value reported came from an average of 5
measurements. The samples were thermostated in a closed stirring vessel before the
surface tension measurements.

Figure 1 shows the obtained behaviour for the different systems analysed in
present work about the influence of composition upon the surface tension value. A
different behaviour is obtained taken into account the amine type. For systems with
DEA and TEA, an increase in NMP concentration in the mixture, produces a
decrease in the value of surface tension. On the other hand, an increase in the value
of this physical property is observed when NMP concentration increases for systems
with MIPA and DIPA.

\newpage
\begin{figure}[h]
 \centerline{\scalebox{0.2}{{\includegraphics{./graphics/garcia1.eps}}}}
 \caption[]{ Influence of mixtures composition upon surface tension. T = 50ºC}\label{figure 1}
\end{figure}
In relation with the surface tension deviation, Figure 2 shows an example of
the calculated values. For all the mixtures analyzed in present work, negative
deviations have been found. This behavior is in agreement with previous studies that
have employed amine-based systems [4].
\\
\begin{figure}[h]
 \centerline{\scalebox{0.2}{{\includegraphics{./graphics/garcia2.eps}}}}
 \caption[]{Influence of mixtures composition upon surface tension deviations. T = 20ºC.}\label{figure 1}
\end{figure}
\\
{\footnotesize
[1] Noll, O.; Fischer, K.; Gmehling, J. J. Chem. Eng. Data, 41, 1434-1438 (1996).
\newline
[2] Fischer, K.; Gmehling, J. Fluid Phase Equilib. 119, 113-130 (1996).
\newline
[3] Thitakamol, B.; Veawab, A.; Aroonwilas, A. Int. J. Greenhouse Gas Control, 1, 318-342 (2007).
\newline
[4] Gómez-Díaz, D; Navaza, J. M. J. Chem. Eng. Data, 49, 1406-1409 (2004).
}

\newpage
\setcounter{figure}{0}
%{\huge \textbf{C. Fong}}
\section*{}
%\section*{P-2~~~~ C. Fong Padrón}
\addcontentsline{toc}{section}{P-2~~~~ C. Fong Padrón}
\begin{center}
{\bf \Large
Intermolecular forces and structures in ionic liquids
}
\\
\vspace{0.5cm}
\underline{Ceila Fong Padrón}, Jesus Rodríguez Otero, and Enrique Cabaleiro Lago
\\
\vspace{0.5cm}
{\it
  Department of Physical Chemistry, University of Santiago de Compostela, Spain
}
\\
\vspace{0.5cm}
{\it E-mail: ceilafong06@gmail.com }
\\
\vspace{0.5cm}
\end{center}
Many physicochemical properties of ionic liquids are now well-characterized and
available from public databases. However, the understanding of their molecular and
electronic structure is still a great challenge and a necessary step to explain their
interesting and unusual properties. Whit this purpose, in this work we decompose the
total interaction energy of seventeen ionic liquids and one ion pair of NaCl by the
Symmetry Adapted Perturbational Theory (SAPT). The ion pair of NaCl is used to
understand how far or near can be the ionic liquid with respect to this classic ion pair.
The seventeen calculated ionic liquids were {\bf m2imBF4}, {\bf m3imBF4}, {\bf emimBF4},
{\bf pmimBF4}, {\bf bmimBF4}, {\bf m2imPF6}, {\bf m3imPF6}, {\bf emimPF6}, {\bf pmimPF6}, {\bf bmimPF6},
{\bf m2imClO4}, {\bf m3imClO4}, {\bf emimClO4}, {\bf pmimClO4}, {\bf bmimClO4} (with these fifteen
compounds the cation effect was studied), {\bf m2imNO3} and {\bf m2imCF3COO} (with these
two compounds and {\bf m2imBF4}, {\bf m2imPF6}, and {\bf m2imClO4} the anion effect was studied).
The calculation was carried out with the Molpro quantum chemistry Package 2009.11
and its implementation for DFT-SAPT calculations. The functionals used were the
combination of the nonempirical GGA, PBE (Perdew-Burke-Ernzerhof) functional as
exchange functional and the local PW91 (Perdew/Wang 91) as correlation functional.
The geometry was obtained previously by optimization at the M06HF/aug-cc-pvdz level
(we have already studied M06HF functional with good results).
\\
\vspace{1.5cm}

{\footnotesize
[1] MOLPRO is a package of ab initio programs written by H.-J. Werner; P. J. Knowles et al.}


\newpage
\setcounter{figure}{0}
\begin{center}
%{\huge \textbf{I. Lorenzo}}
\section*{}
%\section*{P-3~~~~ I. Lorenzo Geada}
\addcontentsline{toc}{section}{P-3~~~~ I. Lorenzo Geada}
{\bf \Large
Stacking interactions between adenine and gallic acid
}
\\
\vspace{0.5cm}
\underline{Isidro Lorenzo}, Nicolás Otero-Martínez, Laura Estévez, and Ana M. Graña
\\
\vspace{0.5cm}
{\it
Departamento de Química Física, Universidade de Vigo, Lagoas-Marcosende s/n 36310-Vigo
Galicia, Spain
}
\\
\vspace{0.5cm}
{\it E-mail: ilorenzo@uvigo.es}
\\
\vspace{0.5cm}
\end{center}
Aromatic $\pi$-stacking interactions play a decisive role in chemistry and
biology. They are fundamental for the geometry characteristics and stabilization
energy of many compounds and mechanisms such as: DNA molecules, tertiary
structure of proteins, etc. Therefore, in order to consider the dispersion that stabilize
them, high-level correlated methods with large basis sets or quantum Monte Carlo
methods obtain significant results, but they are only applicable to the smallest
complexes because of its high computational cost.

Previous studies led us to select MPW1B95 as workhorse functional to carry
out the best results in nonbonded interactions [1] and 6-311++G(2d,2p) 6d basis set.
The obtained wave functions were used to perform QTAIM charge density analysis
with the AIMPAC package of programs [2] and AIM2000 [3].

Parameters such as geometry, charge transfer (CT) and number of critical
points were used to find out the main factors or stabilization. The dimer system
studied is formed by gallic acid, the model system for plant polyphenols and adenine
a DNA base.
\\
\\
a) I\hspace{3.3cm} b) II\hspace{5.5cm} c) III
\begin{figure}[h]
 {\scalebox{0.32}{{\includegraphics{./graphics/lorenzo1.eps}}}}
 {\scalebox{0.35}{{\includegraphics{./graphics/lorenzo2.eps}}}}
 {\scalebox{0.16}{{\includegraphics{./graphics/lorenzo3.eps}}}}
 \caption[]{ Topologies of dimers: I(a), II(b), III(c) }\label{figure 1}
\end{figure}
\\
Three different full optimized structures of the dimer (Figure 1) were obtained
in order to compare stabilization factors such as the energy stability.
The ranges of plane distances, intermolecular distances and tilt angles found,
led us to think: 1) the distance between planes in dimer I plays an important role in
terms of stabilization; 2) in dimer II, one can observe a short distance between the
nearest atoms, and the big tilt angle, due to think that other factors like BCP’s or
other kind of electronic density are implicated in stabilization; 3) there is no pattern
observed in terms of charge transfer.
\newpage
\begin{figure}[h]
 {\scalebox{0.32}{{\includegraphics{./graphics/lorenzo4.eps}}}}
\end{figure}
{\footnotesize
\noindent
[1] Zhao, Y.; Truhlar, D. G. J. Phys. Chem., 109, 5656-5667, 2005.
\newline
[2]. Bader, R. F. W. AIMPAC: A Suite of Programs for the Theory of Atoms in
Molecules; Mc Master University: Hamilton, Canada, 1994.
\newline
[3] Biegler-König, F. W.; Schönbohm, J.; Bayles, D. J. Comp. Chem.
}

\newpage
\setcounter{figure}{0}
%{\huge \textbf{E. I. Martín2}}
\section*{}
%\section*{P-4~~~~ E. I. Martín Fernández}
\addcontentsline{toc}{section}{P-4~~~~ E. I. Martín Fernández}
\begin{center}
{\bf \Large
Phthalocyanines in water: Solvent dynamic properties
}
\\
\vspace{0.5cm}
\underline{Elisa I. Martín}, José Manuel Martínez, and Enrique Sánchez Marcos
\\
\vspace{0.5cm}
{\it
Departamento de Química Física, Universidad de Sevilla
}
\\
\vspace{0.5cm}
{\it E-mail: elisamf@us.es}
\\
\vspace{0.5cm}
\end{center}
Among many important technological applications, Phthalocyanines (Pcs) have been recently identified as promising PSs for PDT [1].

An \textit{ab initio} interaction potential for the system CuPc-H$_{2}$O based in quantum-chemical calculations has been developed and its transferability to the H$_{2}$Pc and [CuPc(SO$_{3}$)$_{4}$]$^{4-}$ cases studied.  Potentials have been tested by means of Molecular Dynamic simulations of CuPc (copper(II) phthalocyanine), H$_{2}$Pc (phthalocyanine) and [CuPc(SO$_{3}$)$_{4}$]$^{4-}$ (copper(II) tetrasulphonate phthalocyanine) complexes in water (represented by the SPC/E model) in order to understand their differential behavior through the analysis of structural, energetical and dynamic properties. The inclusion of the Cu$^{2+}$ cation in the Pc structure reinforces the appearance of two axial water molecules together with a better defined second-shell solvent structure. The presence of SO$_{3}^{-}$ anions implies a well defined hydration shell of about eight water molecules around them making the macrocycle soluble in water [2].  Solvent dynamic properties like mean residence and reorientational times, mean square displacement and lifetime of hydrogen-bonds have been obtained for the first and second solvation shells of the three complexes (red and yellow water respectively in Fig. 1). These results show us that the water molecules of the first shells present larger mean residence and reorientational times than second one, being the mean square displacement of the first water shell smaller than the second one. That fact makes that the lifetime of hydrogen-bonds decrease from first to second shell.
\begin{figure}[h]
 \centerline{\scalebox{0.35}{{\includegraphics{./graphics/martin_poster1.eps}}}}
 \caption[]{ }
\end{figure}
\\
\vspace{0.5cm}
\\
{\footnotesize
[1] I. J. Macdonald, and T. J. Dougherty, \textit{J. Porphyrins Phthalocyanines}, 5, 105 (2001).
\newline
[2] E. I. Martín, J. M. Martínez, and E. Sánchez Marcos, J. Chem. Phys., 134, 024503 (2011).
\newline
}

\newpage
\setcounter{figure}{0}
%{\huge \textbf{A. Sánchez2}}
\section*{}
%\section*{P-5~~~~ A. Sánchez Coronilla}
\addcontentsline{toc}{section}{P-5~~~~ A. Sánchez Coronilla}
\begin{center}
{\bf \Large
Experimental and theoretical study on nucleophilic
additions to tetrabromorhodamine 123
}
\\
\vspace{0.5cm}
\underline{A. Sánchez-Coronilla}, J. A. B. Ferreira, and S. M. B. Costa
\\
\vspace{0.5cm}
{\it
Centro de Química Estrutural, Complexo I, Instituto Superior Técnico
Av. Rovisco Pais, 1049-001 Lisboa. Portugal
}
\\
\vspace{0.5cm}
{\it E-mail: antonio.coronilla@ist.utl.pt}
\\
\vspace{0.5cm}
\end{center}
Rhodamines belong to a widely known class of dyes used nowadays as
tracers in biomaterials.

In this contribution we present a comparative experimental and theoretical
study on the addition reaction of water with tetrabromorhodamine 123 depicted in
Scheme 1. Results of calculation of reactant and transition-state energies were
obtained with different basis sets using DFT [1]. Estimates of the reaction barrier are
in agreement with the experimental activation energy. E$_{a}$ = 18 kcal/mol compares
well with a barrier height of 15 kcal/mol at B3LYP/6-311++G(d,p) level of theory.
To simulate the experimental conditions provided by water and also found in the
presence of other nucleophiles, the solvent effect was included using the integral
equation formalism (IEF) version of the polarizable continuum model (PCM).
\\
\renewcommand{\figurename}{Scheme}
\begin{figure}[h]
 \centerline{\scalebox{0.35}{{\includegraphics{./graphics/sanchez_poster1.eps}}}}
 \caption[]{}\label{figure 1}
\end{figure}
\renewcommand{\figurename}{Figure}
\\
{\footnotesize
[1] R. G. Parr and W. Yang, Density Functional Theory of Atoms and Molecules, Oxford University
Press, New York (1989).
\newline
Acknowledgment. F. C. T. (Portugal): A. Sánchez-Coronilla thanks SFRH/BPD/64898/2009 grant; PTDC/QUI/64658/2006.
}


\newpage
\setcounter{figure}{0}
%{\huge \textbf{A. Sánchez3}}
\section*{}
%\section*{P-6~~~~ A. Sánchez Coronilla}
\addcontentsline{toc}{section}{P-6~~~~ A. Sánchez Coronilla}
\begin{center}
{ \bf \Large
On the addition via C-2 or C-4 of the oxazol-5-one to
nitrostyrenes
}
\\
\vspace{0.5cm}
\underline{A. Sánchez-Coronilla}$^{1}$, R. Rios$^{2}$, and E. I. Martín$^{3}$
\\
\vspace{0.5cm}
{\it
$^{1}$ Centro de Química Estrutural, Complexo I, Instituto Superior Técnico, Avenida Rovisco Pais,
1049-001 Lisboa. Portugal

$^{2}$ ICREA Researcher Departament de Química Orgánica. Universitat de Barcelona
C. Marti i Franques, 1-11, 08028 Barcelona. España

$^{3}$ Departamento de Química Física. Facultad de Química. Universidad de Sevilla
Profesor García González s/n, 41012 Sevilla. España
}
\\
\vspace{0.5cm}
{\it E-mail: antonio.coronilla@ist.utl.pt}
\\
\vspace{0.5cm}
\end{center}
The stereocontrolled construction of quaternary stereocenters is one of the
most difficult challenges for synthetic chemists nowadays. The use of oxazol-5-ones
as nucleophilic reactants in Michael additions has hitherto hardly been studied. Since
oxazol-5-one anions are ambifunctional they can react with activated electrophilic
compounds either at C-2, at C-4 or at the exocyclic oxygen. Recent experimental
studies [1, 2] show the results of the attack at C-2 and C-4 centers in some
oxazol-5-one derivatives. On the basis of these experimental remarks, to know the most stable
compound we have performed for comparative purposes DFT calculations with
different functionals and basis sets of the reactions leading to the C-2 and the C-4
addition compounds (Scheme 1) supposing both of them are formed in each oxazol-5-one derivative. Also studies on the reaction path to get the transition state
conducing to the C-2 and C-4 addition compounds have been performed. The results
of this study show that the most stable compound agree with that compound obtained
experimentally with highest yield and it is the kinetically favoured.
\\
\renewcommand{\figurename}{Scheme}
\begin{figure}[h]
 \centerline{\scalebox{0.4}{{\includegraphics{./graphics/sanchez_poster2.eps}}}}
 \caption[]{Addition at C-2 or C-4 of the oxazol-5-one}
\end{figure}
\renewcommand{\figurename}{Figure}
\\
{\footnotesize
[1] A. Balaguer, X. Companyó, T. Calvet, M. Font-Bardía, A. Moyano and R. Rios. Eur. J. Org.
Chem., 199–203 (2009).
\newline
[2] J. Alemán, A. Milelli, S. Cabrera, E. Reyes and K. A. Jørgensen, Chem. Eur. J., 14, 10958-10966
(2008).
\newline
Acknowledgments. F. C. T. (Portugal): A. Sánchez-Coronilla thanks SFRH/BPD/64898/2009 grant.
}


\newpage
\setcounter{figure}{0}
%{\huge \textbf{M. Varela}}
%\section*{P-7~~~~ M. Varela}
\section*{}
\addcontentsline{toc}{section}{P-7~~~~ M. Varela}
\begin{center}
{\bf \Large
Jet cooled rotational studies of dipeptides
}
\\
\vspace{0.5cm}
\underline{M. Varela}, C. Cabezas, S. Mata, J. C. López, and J. L. Alonso
\\
\vspace{0.5cm}
{\it
Grupo de Espectroscopía Molecular (GEM) Edificio Quifima, Área Química Física. Campus Miguel
Delibes. Parque Tecnológico. Univesidad de Valladolid. E-47005
Valladolid. Spain
}
\\
\vspace{0.5cm}
{\it E-mail: marcelino.varela@uva.es}
\\
\vspace{0.5cm}
\end{center}
Laser ablation molecular beam Fourier transform microwave (LA-MB-FTMW) [1]
spectroscopy, considered the most definitive gas phase structural probe, can
distinguish between different conformational structures since they have unique
spectroscopic constants and give separate rotational spectra. This technique has been
successfully used to study the conformational landscape of the natural aminoacids
[2], their microsolvates [3] and the nucleic acid bases [4]. In this work, we present
the investigation of the conformational preferences of the Gly-Pro dipeptide. Three
conformers have been conclusively identified in the supersonic expansion by the
comparison of the experimental rotational and $^{14}$N(I=1) nuclear quadrupole coupling
constants with those predicted by \textit{ab initio} methods. The quadrupole hyperfine
structure of two $^{14}$N nuclei has been totally resolved and it allows to experimentally
characterize the main intramolecular forces (N--H$\cdots$O=C) which stabilize the assigned
conformers.
\\
\begin{figure}[h]
{\scalebox{1.25}{{\includegraphics{./graphics/varela1.eps}}}}
{\scalebox{1.25}{{\includegraphics{./graphics/varela2.eps}}}}
{\scalebox{1.25}{{\includegraphics{./graphics/varela3.eps}}}}
 \caption[]{The structures of the three conformers identified for Gly-Pro}
\end{figure}
\\
{\footnotesize
[1] Alonso, J. L.; Pérez, C.; Sanz, M. E.; López, J. C.; Blanco, S. Phys. Chem. Chem. Phys., 11, 617,
2008, and references therein.
\newline
[2] (a) S. Mata, V. Vaquero, C. Cabezas, I. Peña, C. Pérez, J. C. López, J. L. Alonso. Phys. Chem.
Chem. Phys., 2009, 11, 4141-4144. (b) S. Blanco, M. E. Sanz, J. C. López, J. L. Alonso, Proc. Natl.
Acad. Sci. USA, 104, 20183-20188 (2007). (c) Sanz, M. Eugenia; López, Juan C.; Alonso, José L.
Phys. Chem. Chem. Phys. (2010), 12(14), 3573-3578.
\newline
[3] J. L. Alonso, E. J. Cocinero, A. Lesarri, M. E. Sanz and J.  C. López. Angew. Chem. Int. Ed, 45, 34713474 (2006).
\newline
[4] (a) V. Vaquero, M. E. Sanz, J.C. López, J. L. Alonso, J. Phys. Chem. A.,111, 3443 (2007). (b) J.C.
López, M. I. Peña, M. E. Sanz, J. L. Alonso, J. Chem. Phys., 126, 191103 (2007). (c) J. L. Alonso, I. Peña,
J. C. López, V. Vaquero, Angew. Chem. Int. Ed. 49, 6141-6143 (2009).
}

\newpage
\setcounter{figure}{0}
%{\huge \textbf{L. Piñeiro}}
\section*{}
%\section*{P-8~~~~ L. Piñeiro Maseda}
\addcontentsline{toc}{section}{P-8~~~~ L. Piñeiro Maseda}
\begin{center}
{\bf \Large
Dye-exchange in micellar systems studied
by fluorescence correlation spectroscopy
}
\\
\vspace{0.5cm}
\underline{Lucas Piñeiro}, Jorge Bordello, Wajih Al-Soufi, and Mercedes Novo
\\
\vspace{0.5cm}
{\it
Universidad de Santiago de Compostela, Facultade de Ciencias
Departamento de Química Física, E-27002 Lugo (Spain)
}
\\
\vspace{0.5cm}
{\it E-mail: lucas.pineiro@rai.usc.es}
\vspace{0.5cm}
\end{center}
The study of exchange dynamics through biological membranes is essential
to understand the transfer of matter in living systems. As model systems micelles are
much simpler than membranes, with well-defined sizes and advantageous optical
properties. Surfactant solutions are highly dynamic systems, where micelles (host)
are in equilibrium with free surfactant and both surfactant and dye molecules (guest)
are being constantly exchanged between the micelles and the surrounding solution.
In the presence of micelles, the dye is distributed between the aqueous solution and
the micellar pseudo-phase [1]. The fast entry and exit of the dye in and out of the
micelle causes fluctuations in the fluorescence intensity, which can be measured
using Fluorescence Spectroscopy Correlation (FCS).

FCS is a fluctuation correlation method that extracts information about the
dynamics of molecular processes from the small changes in molecular concentration
or chemical states that arise from spontaneous fluctuations around equilibrium.
Therefore FCS yields direct information about the dynamics of a fluorescent probe
without the need for external disturbances. FCS allows one to study dynamic and
photophysical processes that take place in a wide time scale in one and the same
experiment. FCS is a single molecule technique, using very small sample volumes
determined by a confocal setup and nanomolar fluorophore concentrations [2].

We investigated the dye-exchange dynamics between different types of
surfactant agents and several dyes with different hydrophobicity which defines the
exchange equilibrium between the phases. From the dynamic and diffusional
properties we obtain parameters which permit us to characterize the interaction and
the distribution of a guest in the different phases.
\\
\vspace{0.5cm}
\\
{\footnotesize
[1] Al-Soufi, W.; Reija, B.; Felekyan, S.; Seidel, C. A.; Novo, M., Chem. Phys. Chem., 9, 1819--1827
(2008).
\newline
[2] Novo, M.; Felekyan, S.; Seidel, C. A. M.; Al-Soufi, J. Phys. Chem. B, 111, 3614--3624 (2007).
}

\newpage
\setcounter{figure}{0}
%{\huge \textbf{A. Vázquez-Carpentier}}
\section*{}
%\section*{P-9~~~~ A. Vázquez Carpentier and M. Martínez Valado}
\addcontentsline{toc}{section}{P-9~~~~ A. Vázquez Carpentier and M. Martínez Valado}
\begin{center}
{\bf \Large
Analysis of an atom laser based on the spatial control of the scattering
length
}
\\
\vspace{0.5cm}
\underline{Alicia Vázquez-Carpentier}$^{1}$, Humberto Michinel$^{1}$, María I. Rodas-Verde$^{1}$, and Víctor M.
Pérez-García$^{2}$
\\
\vspace{0.5cm}
{\it
$^{1}$ Área de Óptica, Facultade de Ciencias de Ourense, Universidade de Vigo, As Lagoas s/n,
Ourense, ES-32004 Spain

$^{2}$ Departamento de Matemáticas, E. T. S. I. Industriales, Universidad de Castilla-La Mancha,
13071 Ciudad Real, Spain
}
\\
\vspace{0.5cm}
{\it E-mail: avcarpentier@uvigo.es}
\\
\vspace{0.5cm}
\end{center}
We propose a simple scheme for constructing an atomic soliton interferometer. In our proposal, a
spatial variation of the scattering length is obtained by means of a laser beam [1] which is also used
to trap an atomic cloud. As a result of the optical control of the atomic interactions, the borders of
the trap lie in a region of negative scattering length whereas the central part of the beam stands in
a zone where the atomic interactions are repulsive. Thus, for a critical number of atoms,
a symmetric emission of counterpropagating pairs of atomic solitons [2] can be produced. The addition
of a parabolic trap makes possible to recombine both beams after their propagation for a given
time, which depends on the initial conditions for soliton emission. The final interference pattern
obtained with the recombination of both solitons contains the information of the phase difference
accumulated by both beams during their propagation along the two arms of the interferometer.
In a second section we describe an alternative method for producing symmetric emission of
counterpropagating pairs of atomic solitons. This method consists in varying the scattering length in
the whole space. Firstly we will make it positive provoking the spread of the BEC. After that, by
turning it negative, a train of pairs of symmetrical solitons will be produced.
\\
\vspace{0.5cm}
\\
{\footnotesize
[1] M. I. Rodas-Verde, H. Michinel and V. M. Pérez-García, Phys. Rev. Lett. 95, 153903 (2005).
\newline
[2] K. E. Strecker, G. B. Partridge, A. G. Truscott and R. G. Hulet, Nature 417, 150 (2002).
}

\newpage
\setcounter{figure}{0}
%{\huge \textbf{D. Josa}}
\section*{}
%\section*{P-10~~ D. Josa}
\addcontentsline{toc}{section}{P-10~~ D. Josa}
\begin{center}
{\bf \Large
Evaluating of Substituent Effects in Corannulene Dimer
}
\\
\vspace{0.5cm}
\underline{D. Josa}, J. Rodríguez-Otero, and E. Cabaleiro-Lago
\\
\vspace{0.5cm}
{\it
 Departamento de Química Física, Facultad de Química, Universidad de Santiago de Compostela, 15782
                                      Santiago de Compostela, Spain
}
\\
\vspace{0.5cm}
{\it E-mail: danijosa@gmail.com }
\\
\vspace{0.5cm}
\end{center}
          Progress and numerous applications in fullerene chemistry have sparked a
special interest in the chemistry of curved polycyclic aromatic hidrocarbons (1).
Computacional studies of $\pi$-$\pi$ interactions in planar aromatic systems have been well
documented in literature (2,3). However, very few studies addressing $\pi$-$\pi$ interactions in
curved conjugated systems have been published (1,4). The aim of this work is to
investigate the substituent effects in corannulene dimer. The introduction of substituents
in the structure of corannulenes is of great interest because may exert a modulatory
effect and enable a better concave-convexe interaction with fullerenes.The corannulene
monomers were optimized at the B3LYP/6-311G* level. Figure 1 shows one of the
dimers investigated. The interaction energy and equilibrium distance between the
monomers were calculated using $\omega$B97XD, B97D and M06-2X funcionals and 6-31+G*
basis sets. Counterpoise corrections were applied to all reported interaction energies (5).
The substituent effects were evaluated based on the contribution of dispersion and
molecular electrostatic maps.
\begin{figure}[h]
 \centerline{\scalebox{0.8}{{\includegraphics{./graphics/josa1.eps}}}}
 \caption[]{(left) A dimer formed by “C$_{60}$-like” (with the curvature of buckmisterfullerene,C$_{60}$) and corannulene-5CN monomers. (right) Substituted corannulene monomer studied.  }\label{figure 1}
\end{figure}
\\
{\footnotesize
[1] A. Sygula, and S. Saebo. Int. J. Quantum. Chem.,109, 65, (2009).
\newline
[2] M. O. Sinnokrot, and C. D. Sherrill. Int. J. Phys. Chem. A.,110, 10656, (2006).
\newline
[3] O. I. Obolensky, V. V. Semenikhina, A. V. Solov'yov, and W. Greier. Int. J.Quantum. Chem., 107, 1335, (2007).
\newline
[4] S. Tsuzuki, T. Uchimaru, and K. Tanabe. J. Phys.Chem A.,102, 740, (1998).
\newline
[5] S. F. Boys and F. Bernardi, Mol. Phys., 19, 353, (1970).
}
\newpage
